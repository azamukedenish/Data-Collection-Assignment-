\documentclass[options]{article}

 \usepackage[
    top    = 2.75cm,
    bottom = 2.50cm,
    left   = 4.00cm,
    right  = 3.50cm]{geometry}

\usepackage[parfill]{parskip}
\pagenumbering{roman}
\title{Agriculture! Should it still be a prerequisite to pursue Bachelor of Science in Agriculture?}
\author{Azamuke Denish\thanks{supervisor: Ernest Mwebaze}}\newpage
\date{%
    Makerere University\\%
    Feb 21, 2018
}


\begin{document}
\begin{titlepage}
\maketitle
\end{titlepage}




\newpage
\pagenumbering{arabic} 
\section{\textbf{ Introduction}} 
Agriculture is often an interesting field in a community. This is probably due to the  daily benefits that communities have been able to realize from it. As you are perhaps aware, agricultural science  is not simply the study of crops.


\subsection{\textbf{Background}}
Agriculture is the science and practice of farming, including cultivation of the soil for the growing of crops and the rearing of animals to provide food, fiber, and other products. Agriculture covers six fields of study, namely: crop science; animal science; soil science; crop protection; economics and marketing; and agricultural extension and communication. Agricultural scientists study farm animals, crops and factors affecting farm production, to improve the efficiency and sustainability of farms and related agricultural enterprises. They collect and analyze samples of produce, feed, soil, water and other elements that may be affecting agricultural production.
\bigbreak

Agricultural science can be thought of as the study of crop production and animal rearing. While studying agriculture, it becomes apparent that the theory and practicals go hand in hand. In the bachelor of science in agriculture curriculum, students are introduced to various agricultural concepts such as animal agriculture, food security and nutrition, crop diseases, animal health and hygiene , apiculture, principles of farm management and accounts. \bigbreak

This explains why passing agriculture at the lower levels is a pre-requisite before one can study bachelor of science in agriculture in any reputable university across the globe. Therefore this gave birth to a belief that all students of bachelor of science in agriculture should at least be having some previous knowledge of agriculture. 


\subsection{\textbf{Problem Statement}}
This project will examine the effects of agricultural background on the academic performance and field skills of a student pursuing bachelor of science in agriculture.


\subsection{\textbf{Objectives}}


\subsubsection{\textbf{Main Objective}} 
The main goal of this project is to determine whether one needs to have Agricultural background in order to excel in bachelor of science in agriculture. 


\subsubsection{\textbf{Specific Objectives}}

\begin{itemize}
  \item To collect all the data necessary to aid our research.
  \item To perform a thorough analysis on the collected data.
  \item To come up with a conclusion from the data analysis.
\end{itemize}


\subsection{\textbf{Scope}}
This research is aimed at students offering bachelor of science in agriculture at higher institutions of learning such as Makerere University.

\subsection{\textbf{Research Significance}}
This study is useful because it aims at improving the curriculum of bachelor of science in agriculture at higher institutions of learning.



\section{\textbf{Literature review}}

I looked at \cite{latexGuide}Makerere University's B.Sc. in Agriculture program, and for a candidate to be admitted, he/she must have: At least a subsidiary pass in Agriculture in the Uganda Advanced Certificate of Education (UACE) or its equivalent and at least two principal passes at the same sitting in UACE in any of the following subjects: - Biology,
Chemistry, Geography, Physics.


\section{\textbf{Methodology}}
The proposed methodology consists of two phases, data collection and data analysis.\bigbreak
Data will be collected using ODK Collect, which will later on be uploaded to the ODK aggregate server to carry out all the required analysis. Different kinds of data (including images and GPS coordinates) will be collected, these include: 

\begin{itemize}
  \item Student’s name, registration number, gender, recent photo and place of residence (including GPS coordinates)
  \item UACE grade for Agriculture
  \item Current performance at the university (CGPA). 
\end{itemize}



\begin{thebibliography}{10} \bibitem{latexGuide} Makerere University, \emph{Bachelor of science in Agriculture}, Available at \texttt{https://courses.mak.ac.ug/programmes/bachelor-science-agriculture} \end{thebibliography}



\end{document}